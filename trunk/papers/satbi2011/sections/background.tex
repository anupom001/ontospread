In this section, the theoretical model of \textit{SA}~\cite{Collins_Loftus_1975,Scott1981} is reviewed to 
illustrate the basic components and the operations performed by SA during their execution, specially
the spreading of the activation from a node to their adjacent nodes. This model is made up of a 
conceptual network of nodes connected through relations (conceptual graph). Taking into account 
that nodes represent domain objects or classes and edges relations among them, it is possible to 
establish a semantic network in which SA can be applied. The process performed by the algorithm 
is based on a thorough method to go down the graph using an iterative model. Each iteration is comprised of 
a set of beats, a stepwise method, and the checking of a stop condition. SA is comprised of three stages: 
\textit{Preadjustement} and \textit{Postadjustement} that are usually in charge of performing 
some control strategy over the target semantic network and the set set of activated concepts and the \textit{Spreading} 
stage in which concepts are activated in activation waves. The calculation of the activation rank $I_i$ of a node $n_i$ is 
defined as follows:

\begin{equation}
I_i  = \sum_j{O_j \omega_{ji}}
\end{equation}
\medskip

$I_i$ is the total inputs of the node $n_i$, $O_j$
is the output of the node $n_j$ connected to $n_i$ and $\omega_{ji}$
is the weight of the relation between $n_j$ and $n_i$. 
If there is not relation between $n_j$ and $n_i$ then
$\omega_{ji} = 0$. 

The activation function $f$ is used to evaluate the ``weight'' of a node and
decide if the concept is active.

\begin{equation}
N_i=f(I_i)=\begin{cases} 0 & \text{if $I_i < \jmath_i$} \\ 1 &
\text{if $I_i > \jmath_i$}
\\ \end{cases}
\end{equation}

$N_i$ is $1$ if the node has been activated or 0 otherwise. 
$\jmath_i$, the threshold activation value for node $i$, depends on the application
and it can change from a node to others. The activation rank $I_i$ of a
node $n_i$ will change while algorithm iterates.


