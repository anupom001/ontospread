%Digital Libraries: description and needs
The improvements in digitization lead us to a new environment in which digital libraries and archives are designed and used
in a new way. This situation implies new challenges in the digital formats, storage (information is continuously growing) and
information retrieval models. Following the recommendations of the European Commision~\cite{DigitalLibraries} the digital libraries are
a key factor to bring out the full economic and cultural potential of Europe’s cultural and scientific heritage 
through the Internet. The online presence of material from different cultures and in different languages will
make it easier for citizens to appreciate their own cultural heritage as well as the heritage
of other European countries. Besides its fundamental cultural value, cultural material is an important resource for new
added value services. That is why more sophisticated software tools and methods are needed to meet the expectations of users easing
the information retrieval of these large datasets and overcoming the classical problems of information overloading. 

Initiatives like Semantic Web and Linked Data~\cite{LinkedData} tries to define vocabularies and ontologies enabling
the data interoperability and sharing that enable by means of the Web infrastructure the access
to the contents across different platforms and applications. The development of tools using these common
data formats and models is largely implemented and representative to digital libraries
but some algorithms and methods are not yet promoted to work with them in a standard way preventing 
the improvement and effectiveness of information access. 

In this sense Spreading Activation technique (hereafter SA) introduced by~\cite{Collins_Loftus_1975}, in the field of 
psycho linguistics and semantic priming, purpose a model in which all relevant information is mapped
on a graph as nodes with a certain ''activation value``. Relations between two concepts
are represented by a weighted edge. If a node is activated their activation value is spread
to their neighbor nodes. This technique was adopted by the computer science community and applied
to the resolution of different problems, see Sect.~\ref{related-work}. In the field of digital libraries
this technique can ease the information access providing a connectionist method to retrieve data like
brain can do. Although SA is widely used, more specifically in recent years have been successfully applied 
to ontologies, a common and standard API is missing and each third party interested in their application
 must to implement its own version~\cite{SpreadingLarkc} of SA.

Taking into account this new information realm and the leading features of putting together 
the SA and the Semantic Web technologies, new enriched services of searching, 
matchmaking, recommendation or contextualization can be implemented to fulfill the requirements
of access information in digital libraries of different trending scopes like e-procurement, 
legal document databases~\cite{bopaEstonia}, e-tourism or e-health.

\subsection{Main Contributions}
The proposed work aims to afford a framework for SA to ease the configuration, customization 
and execution of them over semantic networks and more specifically over RDF graphs and ontologies. It is relevant
to digital libraries access and interoperability due to the fact that this technique provides
a set of proven algorithms for retrieving and recommending information resources in large knowledge bases. 
Following the specific contributions of this work are listed:
\begin{itemize}
 \item Study and revision of the classical constrained SA.
 \item Study and definition of new restrictions for SA applied to ontologies and RDF graphs.
 \item Implementation of a whole and extensible framework (called ONTOSPREAD API).
to customize and perform the SA based on good practices in software programming.
\item Outlining of a methodology to configure and refine the execution of SA.
\item An example of configuration and refinement applying SA over a well-known ontology, the Galen ontology.
\end{itemize}


%FIXME: This paper is structured as follows...