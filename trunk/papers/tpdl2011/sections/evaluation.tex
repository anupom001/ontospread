The validation of the algorithm depends on the configuration of the activation and spreading processes to
fit it to the different domain issues. SA is determined by the target semantic network and therefore
the domain knowledge (concepts and relations) defined is the key part to adjust its behavior. On the other
hand, taking into account that the activation and spreading is guided by the weights of relations the specification
of them is fundamental to get the desired outputs. The methodology to test the implementation
of the algorithm is subjected to these conditions but a step-wise refinement method can be outlined:
\begin{enumerate}
  \item Use a semantic network (ontology, etc.) well-known: concepts and relations.
  \item Define potential set of initial concepts ($Q_{sem}$) and their initial activation value (commonly $1.0$).
  \item Specify the weights of the relations to that domain knowledge.
  \item Combine the different restrictions provided by the API.
  \item Select the degradation function.
  \item Add the reward techniques to increase the activation value of certain nodes.
  \item Try to evaluate new activation functions for their further implementation.
  \item Repeat these steps until getting the most appropriated set of output concepts ($Q'_{sem}$)
  to that domain knowledge
\end{enumerate}

To apply this methodology, the Galen Ontology~\footnote{\url{http://www.opengalen.org/}} has been selected. It is the reference ontology in the
biomedicine domain and it is widely used in reasoning and decision support processes. The design of
the experiment depends on: the ontology (the API is not restricted to work with only one ontology),
the weights of relations, the set of initial concepts, the set of restrictions, the degradation function and the extensions to reward nodes. Following these variables
are specified to run the algorithm and get a good refinement of SA in the selected domain. The structure of the Galen ontology 
is summarized in Tab \ref{table:info-ontologia}:

\begin{longtable}[c]{|l|r|} 
\caption{Galen Ontology.}\label{table:info-ontologia}\\    
\hline
  \multicolumn{2}{|c|}{\textbf{Stats of Galen Ontology}}\\ \hline
  \textit{Type} &  \textit{Number} \\\hline

\endhead

    Object Properties&0 \\ \hline    
    Clases&0 \\ \hline
    Enumerated Classes&0 \\ \hline
    Union Classes&0 \\ \hline
    ComplementOf Classes&0 \\ \hline
    Intersection Classes&0 \\ \hline
    Restrictions&0 \\ \hline    
    Instances&0\\ \hline
\end{longtable}

The set of initial concepts ($Q_{sem}$) with an initial value $1.0$ is: ``\#Foo'' and ``\#Bar''. The weights of the 
relations are fixed to a default value of $1.0$. The refinement of the algorithm will enable us to get
a set of output concepts ($Q'_{sem}$) similar to the process that a brain will do. The degradation functions
and the reward technique will be alternatively combined checking the output of the algorithm. Finally, the restrictions 
and their configuration values are listed in Tab.~\ref{tabla:test-restricciones}.

\begin{table}[htb]
\renewcommand{\arraystretch}{1.3}
\begin{center}
\begin{tabular}{|p{8cm}|p{2cm}|}
\hline
        \multicolumn{2}{|c|}{\textbf{Restriction Values}}\\        
        \hline
        \textit{Type} &  \textit{Value} \\ \hline
    Minimum activation value $N_{\min}$ &0 \\ \hline
    Minimum number of spread concepts $\mathbb{M_{\min}}$&0 \\ \hline
    Maximum number of spread concepts $\mathbb{M}$&0 \\ \hline
    Time of activation $t$&0 \\ \hline
    Context of activation $\mathbb{D}_{com}$&DEFAULT (\textit{retries}=0)\\ \hline
   		\hline
		\end{tabular}
		\caption{Restrictions to test SA.}
		\label{tabla:test-restricciones}
  \end{center}
\end{table} 

The execution of the first test get the next results, see Tab.~\ref{tabla:a1}:

\begin{longtable}{|p{10cm}|p{4cm}|}        
        \caption{Results of the first test.}
        \label{tabla:a1}\\ \hline
        \multicolumn{2}{|c|}{\textbf{Output Metrics}}\\
        \hline
        \textit{Type} &  \textit{Value} \\
        \hline
        \endfirsthead
        \caption[]{Results of the first test (continues).}\\
        \hline
        \endhead
        \hline
        \multicolumn{2}{|c|}{$\ldots$}\\
        \hline
        \endfoot
        \hline
        \endlastfoot	    
	                Spread Nodes & $0$ \\ \hline
		        Activated Nodes& $0$ \\ \hline
		        Highest activation value & $0$ \\ \hline
		        Deepest spread path &  $\infty$ \\ \hline
		        Time of activation $t$ & $0$\\ \hline
		        Degradation function &$h_1$\\ \hline
		        Reward &No\\ \hline
         \multicolumn{2}{|c|}{\textbf{$Q'_{sem}$}}\\\hline
           \textit{URI} &  \textit{Value} \\ \hline
	        \#Foo&  $0$ \\ \hline
	         \multicolumn{2}{|c|}{\ldots}\\\hline
     \end{longtable}      

After that the configuration is changed to use the reward of converging paths and adjusting the restrictions with the next values:
FIXME: values. The results of this execution are demonstrated in Tab.~\ref{tabla:a2}

\begin{longtable}{|p{10cm}|p{4cm}|}        
        \caption{Results of the second test.}
        \label{tabla:a2}\\ \hline
        \multicolumn{2}{|c|}{\textbf{Output Metrics}}\\
        \hline
        \textit{Type} &  \textit{Value} \\
        \hline
        \endfirsthead
        \caption[]{Results of the second test (continues).}\\
        \hline
        \endhead
        \hline
        \multicolumn{2}{|c|}{$\ldots$}\\
        \hline
        \endfoot
        \hline
        \endlastfoot	    
	                Spread Nodes & $0$ \\ \hline
		        Activated Nodes& $0$ \\ \hline
		        Highest activation value & $0$ \\ \hline
		        Deepest spread path &  $\infty$ \\ \hline
		        Time of activation $t$ & $0$\\ \hline
		        Degradation function &$h_1$\\ \hline
		        Reward &No\\ \hline
         \multicolumn{2}{|c|}{\textbf{$Q'_{sem}$}}\\\hline
           \textit{URI} &  \textit{Value} \\ \hline
	        \#Foo&  $0$ \\ \hline
	         \multicolumn{2}{|c|}{\ldots}\\\hline
     \end{longtable}    

FIXME: Concluir esultados
