Since SA was introduced by~\cite{Collins_Loftus_1975} in the field of 
psycho linguistics and semantic priming it has been applied to the resolution
of problems trying to simulate the behavior of the brain using a connectionist method
to provide an ``intelligent'' way to retrieve information and data. 

The use of SA was motivated due to the research on graph exploration~\cite{Scott1981,AndersonTheory}. Nevertheless
the success of this technique is specially relevant to the fields of Document~\cite{turtle91inference} 
and Information Retrieval~\cite{Cohen1987}. It has
been also demonstrated its application to extract correlations between query terms and documents analyzing user 
logs~\cite{Cui03} and to retrieve resources amongst multiple systems~\cite{Schumacher+2008search} 
in which ontologies are used to link and annotate resources.

In recent years and regarding the emerging use of ontologies in the Semantic Web area new applications of SA have
appeared to explore concepts~\cite{Qiu93,Chen95} addressing the two important issues: 1) the selection and 2) the weighting of
additional search terms and to measure conceptual similarity~\cite{gouws-vanrooyen-engelbrecht:2010:CCSR}. 
On the other hand, there are works~\cite{DBLP:journals/cogsr/KatiforiVD10} 
exploring the application of the SA on ontologies in order to create context inference models.The 
semi-automatically extension and refinement of ontologies~\cite{liu_et_al_2005} is other trending topic to apply SA
in combination with other techniques based on natural language processing. Data mining,
more specifically mining socio-semantic networks\cite{paper:troussov:2008}, and applications 
to collaborative filtering (community detection based on tag recommendations, expertise location, etc.) are other 
potential scenarios to apply the SA theory due to the high performance and high scalability of the technique.

In particular, annotation and tagging~\cite{labraTagging2007} services to gather 
meta-data~\cite{GelgiVD05} from the Web or to predict social annotation~\cite{Chen:2007:PSA:1780653.1780702} and recommending 
systems based on the combination of ontologies and SA~\cite{citeulike:3779904} are taken advantage of using SA technique. 
Also the semantic search~\cite{conf-sofsem-Suchal08} is a highlight area to apply SA following
hybrid approaches~\cite{bopaEstonia,RochaSA04} or user query expansion~\cite{767402} combining metadata 
and user information.

Although this technique is widely accepted and applied to different fields open implementations\footnote{ 
Texai company~(\url{http://texai.org/}) offers a proprietary implementation of SA.}, are missing. Moreover 
the Apache Mahout~\footnote{\url{http://mahout.apache.org/}} project, a recent scalable machine learning library 
that supports large data sets, does not include an implementation of SA instead of 
providing algorithms for the classification, clustering, pattern mining, 
recommendation and collaborative filtering of resources in which SA should be representative. 
 


