This work provides a configurable and extensible framework to support the SA technique. It allows
the configuration of restrictions and their combination to get
the most accurate set of output concepts. One of the features that
turns SA to a widely accepted algorithm lies in its flexibility 
but some disadvantages are also presented: the adjusting and refinement of
restrictions and weights of the relations, the selection of the
degradation function and the use of reward functions. This framework
minimizes these advantages with an extensible library that can be
applied to different scenarios like digital libraries, in particular
biomedicine, e-procurement, e-health, etc. providing enriched services
of annotation, searching or recommendation.

The main improvement in the algorithm consists on the flexibility of
the refinement methodology. An automatic learning algorithm to create SA configurations
according to ontologies should be developed. Thus, the training stage of SA could generate
the best configuration for a specific domain. The algorithm could optimize the selection 
of input parameters like the weights of the relations, the degradation functions 
or the combination of restrictions. Beside new measures related to instances such as `Cluster
Measure'', ``Specifity Measure'' or both could be used in the process of activation/spreading.
Also the selection of the next node to spread is based on a ``first
better'' strategy (if two nodes have the same activation value) because of this fact
other selection strategies should be implemented. Finally a new version of the SA
is being specified and developed following the Map/Reduce\footnote{\url{http://labs.google.com/papers/mapreduce.html}} programming model with the objective
of getting a distributed version of this technique for processing large data sets.

% \textbf{Acknowledgements}. ONTOSPREAD was initially developed in BOPA~\cite{bopaEstonia} project that is
% one of \textit{Semantic Web Use Cases and Case
% Studies}\footnote{\url{http://www.w3.org/2001/sw/sweo/public/UseCases/CTIC/}}
% collected by W3C. Currently it is being applied to the process of searching
% public procurement notices in the ``10ders Information Services''\footnote{\url{http://rd.10ders.net}} project, partially funded by 
% the Spanish Ministry of Industry, Commerce and Tourism (TSI-020100-2010-919) 
% and the European Regional Development Fund (EFDR), leaded by ``Gateway Strategic Consultancy Services''\footnote{\url{http://gateway-scs.es/}} 
% and developed in cooperation with ``Exis TI''\footnote{\url{http://www.exis-ti.com/}}  and WESO Research Group.