The Spreading Activation technique (hereafter SA) introduced by~\cite{Collins_Loftus_1975}, in the field of 
psycho linguistics and semantic priming, proposes a model in which all relevant information is mapped
on a graph as nodes with a certain ''activation value``. Relations between two concepts
are represented by a weighted edge. If a node is activated their activation value is spread
to their neighbor nodes. This technique was adopted by the computer science community and applied
to the resolution of different problems, see Sect.~\ref{related-work}, and it is relevant to
the the digital libraries field in the scope of: 1) construction of hybrid semantic search engines and 2) ranking of
information resources according to an input set of weighted resources. Thus this technique
eases the information access providing a connectionist method to retrieve data like brain can do. 
Although SA is widely used, more specifically in recent years has been successfully applied 
to ontologies, a common and standard framework is missing and each third party interested in its application
 must to implement its own version~\cite{SpreadingLarkc} of SA.

Taking into account the new information realm and the leading features of putting together 
the SA technique and the Semantic Web and Linked Data initiatives, new enriched services of searching, 
matchmaking, recommendation or contextualization can be implemented to fulfill the requirements
of access information in digital libraries of different trending scopes like legal document databases~\cite{bopaEstonia}, 
e-procurement, e-tourism or e-health.

The proposed work aims to afford a framework for SA to ease the configuration, customization 
and execution over graph-based structures and more specifically over RDF graphs and ontologies. It is relevant
to digital libraries access and interoperability due to the fact that this technique provides
a set of proven algorithms for retrieving and recommending information resources in large knowledge bases. 
Following the specific contributions of this work are listed: 1) study and revision of the classical constrained SA;
2) study and definition of new restrictions for SA applied to RDF graphs and ontologies; 
3) implementation of a whole and extensible framework (called ONTOSPREAD) to customize 
and perform the SA based; 4) outlining of a methodology to configure and refine the execution of SA and 
5) an example of configuration and refinement applying SA over a well-known ontology, the Galen ontology.


%FIXME: This paper is structured as follows...