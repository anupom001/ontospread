The Spreading Activation technique (hereafter SA) introduced by~\cite{Collins_Loftus_1975}, in the field of 
psycho linguistics and semantic priming, proposes a model in which all relevant information is mapped
on a graph as nodes with a certain ''activation value``. Relations between two concepts
are represented by a weighted edge. If a node is activated their activation value is spread
to their neighbor nodes. This technique was adopted by the computer science community and applied
to the resolution of different problems, see Sect.~\ref{related-work}, and it is relevant to
the medical fields sector in the scope of: 1) construction of hybrid semantic search engines;  2) ranking of
information resources according to an input set of weighted resources 3) recommendation of medical terms using
well-known ontologies in a particular sector and 4) decision-support easing the access to the information
in large databases . Thus this technique provides a connectionist method to retrieve data like brain can do. 
Although SA is widely used, more specifically in recent years has been successfully applied 
to ontologies, a common and standard framework is missing and each third party interested in its application
 must to implement its own version~\cite{SpreadingLarkc} of SA.

Taking into account the new information realm and the leading features of putting together 
the SA technique and the Semantic Web and Linked Data initiatives, new enriched services of searching, 
matchmaking, recommendation or contextualization can be implemented to fulfill the requirements
of access information in different trending scopes like e-health, e-procurement, e-tourism or legal 
document databases~\cite{bopaEstonia}. More specifically in the e-health sector there 
is a growing need to automate the processes related to the tagging of electronic clinical records and to create 
tools for the clinical decision support. That is why SA is relevant to the field of clinical knowledge management 
technologies through its capacity to process large databases and exploit the know-how
of previous records easing the recommendations of information resources.


The proposed work aims to provide a framework for SA to ease the configuration, customization 
and execution over graph-based structures and more specifically over RDF graphs and ontologies. It is relevant
to medical systems access and interoperability due to the fact that this technique is based on
a set of proven algorithms for retrieving and recommending information resources in large knowledge bases. 
Following the specific contributions of this work are listed: 1) study and revision of the classical constrained SA;
2) study and definition of new restrictions for SA applied to RDF graphs and ontologies; 
3) implementation of a whole and extensible framework (called ONTOSPREAD) to customize 
and perform the SA based; 4) outlining of a methodology to configure and refine the execution of SA and 
5) an example of configuration and refinement applying SA over two well-known ontologies: GALEN and SNOMED-CT.


%FIXME: This paper is structured as follows...
\subsection{Organization}
This paper is structured as follows: in Section 2, we review the relevant
work in medical systems and the common applications of SA. In Section 3, we provide a description of the design and implementation
of an open framework for SA technique, explaining the algorithm, restrictions, etc. Afterwards, in Section 4 
we apply the ONTOSPREAD framework over the GALEN and SNOMED-CT ontologies to evaluate the SA
technique for recommending concepts in medical systems. Finally, we
evaluate the results of the previous executions and present some conclusions.
